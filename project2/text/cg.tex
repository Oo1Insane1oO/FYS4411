\documentclass[a4paper, hidelinks, 10pt]{article}
\usepackage[utf8]{inputenc}
\usepackage[T1]{fontenc}
\usepackage[expert]{mathdesign}
\usepackage{listings}
\usepackage[pdftex]{graphicx}
\usepackage{color}
\usepackage{mathtools}
\usepackage{amssymb}
\usepackage[margin=0.7in]{geometry}
\usepackage{hyperref}
\usepackage{subcaption}
\usepackage{float}
\usepackage{caption}
\usepackage{algpseudocode}
\usepackage{scrextend}
\usepackage{ragged2e}
\usepackage{multirow}
\usepackage{array}
\usepackage{physics}
\usepackage{enumitem}
\usepackage{varwidth}
\usepackage{tikz}
\usepackage{etoolbox}
\usepackage{fancyhdr}
%\usepackage{fouriernc}

%specific reused text
\newcommand{\mdate}{.}
\newcommand{\mtitle}{.}
\newcommand{\mauthor}{.}
\newcommand{\massignn}{.}

\pagestyle{fancy}
\fancyhf{}
% \fancyhead[LO, RE]{\small\leftmark}
\lhead{\small{\mtitle}}
\chead{\small{\massignn}}
\rhead{\small{\thesection}}
% \lfoot{}
\cfoot{\thepage}
% \rfoot{}

%renew title numbering
\renewcommand{\thesection}{Problem \arabic{section}}
\renewcommand{\thesubsection}{\thesection\alph{subsection}}

%center title and subtitle
\let\oldsection\section
\renewcommand{\section}[1]{\centering \oldsection{{#1}} \justifying}
\let\oldsubsection\subsection
\renewcommand{\subsection}[1]{\centering \oldsubsection{{#1}} \justifying}

%title settings
% \renewcommand{\headrulewidth}{0pt}
\renewcommand{\sectionautorefname}{section}
\renewcommand{\subsectionautorefname}{section}
\renewcommand{\equationautorefname}{equation}
\renewcommand{\figureautorefname}{figure}
\renewcommand{\tableautorefname}{table}
\captionsetup{compatibility=false}

\patchcmd{\smallmatrix}{\thickspace}{\kern1.3em}{}{}

\definecolor{codegreen}{rgb}{0,0.6,0}
\definecolor{codegray}{rgb}{0.3,0.3,0.3}
\definecolor{codepurple}{rgb}{0.58,0,0.82}
\definecolor{backcolour}{rgb}{0.95,0.95,0.92}
\lstdefinestyle{mystyle}{
        backgroundcolor=\color{backcolour},
        commentstyle=\color{codegreen},
        keywordstyle=\color{magenta},
        numberstyle=\tiny\color{codegray},
        stringstyle=\color{codepurple},
        basicstyle=\footnotesize,
        breakatwhitespace=false,
        breaklines=true,
        captionpos=b,
        keepspaces=true,
        numbers=left, 
        numbersep=4pt, 
        showspaces=false, 
        showstringspaces=false,
        showtabs=true, 
        tabsize=2
}
\lstset{style=mystyle}

\hypersetup{
    colorlinks=true,
    linkcolor=black,
    filecolor=magenta,
    urlcolor=blue,
}
\urlstyle{same}

\newcommand{\onefigure}[4]{
    \begin{figure}[H]
        \centering
        \textbf{{#1}}\\
        \includegraphics[scale=0.65]{{#2}}
        \caption{{#3}}
        \label{fig:#4}
    \end{figure}
    \justifying
} %one figure {filename}{caption}
\newcommand{\twofigure}[7]{
    \begin{figure}[H]
        \centering
        \begin{subfigure}[b!]{0.49\textwidth}
            \centering
            \includegraphics[width=\textwidth]{{#1}}
            \caption{{#2}}
            \label{subfig:#3}
        \end{subfigure}
        \begin{subfigure}[b!]{0.49\textwidth}
            \centering
            \includegraphics[width=\textwidth]{{#4}}
            \caption{{#5}}
            \label{subfig:#6}
        \end{subfigure}
        \caption{#7}
        \justify
    \end{figure}
} %two figure one-line {title}{file1}{caption1}{file2}{caption2}

\newcommand{\prtl}{\mathrm{\partial}} %reduce length of partial (less to write)
\NewDocumentCommand{\prd}{m O{} O{}}{\frac{\prtl^{#3}{#2}}{\prtl{#1}^{#3}}}
\newcommand{\vsp}{\vspace{0.2cm}} %small vertical space
\newcommand{\txtit}[1]{\textit{{#1}}} %italic text
\newcommand{\blds}[1]{\boldsymbol{{#1}}} % better bold in mathmode (from amsmath)
\newcommand{\bigO}{\mathcal{O}} %nice big O
\newcommand{\me}{\mathrm{e}} %straight e for exp
\newcommand{\md}{\mathrm{d}} %straight d for differential
\newcommand{\mRe}[1]{\mathrm{Re}\left({#1}\right)}%nice real
\newcommand{\munit}[1]{\;\ensuremath{\, \mathrm{#1}}} %straight units in math
\newcommand{\Rarr}{\Rightarrow} %reduce lenght of Rightarrow (less to write)
\newcommand{\rarr}{\rightarrow} %reduce lenght of rightarrow (less to write)
\newcommand{\ecp}[1]{\left< {#1} \right>} %expected value
\newcommand{\urw}{\uparrow} % up arrow
\newcommand{\drw}{\downarrow} % up arrow
\newcommand{\pt}[1]{\textbf{\txtit{#1}}\justify}
\newcommand{\infint}{\int\limits^{\infty}_{-\infty}}
\newcommand{\oinfint}{\int\limits^{\infty}_0}
\newcommand{\sint}{\int\limits^{2\pi}_0\int\limits^{\pi}_0\oinfint}
\newcommand{\arcsinh}[1]{\text{arcsinh}\left(#1\right)}
\newcommand{\I}{\scalebox{1.2}{$\mathds{1}$}}
\newcommand{\veps}{\varepsilon} %\varepsilon is to long :P

\newcommand\numberthis{\addtocounter{equation}{1}\tag{\theequation}}

\makeatletter
% define a macro \Autoref to allow multiple references to be passed to \autoref
\newcommand\Autoref[1]{\@first@ref#1,@}
\def\@throw@dot#1.#2@{#1}% discard everything after the dot
\def\@set@refname#1{%    % set \@refname to autoefname+s using \getrefbykeydefault
    \edef\@tmp{\getrefbykeydefault{#1}{anchor}{}}%
    \def\@refname{\@nameuse{\expandafter\@throw@dot\@tmp.@autorefname}s}%
}
\def\@first@ref#1,#2{%
  \ifx#2@\autoref{#1}\let\@nextref\@gobble% only one ref, revert to normal \autoref
  \else%
    \@set@refname{#1}%  set \@refname to autoref name
    \@refname~\ref{#1}% add autoefname and first reference
    \let\@nextref\@next@ref% push processing to \@next@ref
  \fi%
  \@nextref#2%
}
\def\@next@ref#1,#2{%
   \ifx#2@ and~\ref{#1}\let\@nextref\@gobble% at end: print and+\ref and stop
   \else, \ref{#1}% print  ,+\ref and continue
   \fi%
   \@nextref#2%
}
\makeatother

\begin{document}
\thispagestyle{empty}
\begin{center} \vspace{1cm}
    \textbf{\Large{\mtitle}}\\ \vspace{0.5cm}
    \textbf{\large{\massignn}}\\ \vspace{1cm}
    \textbf{\large{\mauthor}}\\ \vspace{0.5cm}
    \Large{\mdate}\\ \vfill
\end{center}

\clearpage
\setcounter{page}{1}

\subsection{Analytic Expression for Hessen Matrix}
\label{ssub:analytic_expression_for_hessen_matrix}
\newcommand{\psinorm}{\int\psi_T^2\md\tau}
\newcommand{\psigender}[1]{\prd{c_{#1}}[\psi_T]}
\newcommand{\psialphder}{\prd{\alpha}[\psi_T]}
\newcommand{\psialphsecder}{\prd{\alpha}[\psi_T][2]}
\newcommand{\psibetder}{\prd{\beta}[\psi_T]}
\newcommand{\psibetsecder}{\prd{\beta}[\psi_T][2]}
\newcommand{\gsum}{\sum\limits_{i\neq j} \frac{a}{\left(\beta + \frac{1}{r_{ij}}\right)}}
    The Hessen matrix is needed in order to minimize with the conjugate
    gradient method as described in \Autoref{sub:conjugate_gradient_method}.
    The Hessen matrix in the case where we minimize the local energy with
    respect to $\alpha$ and $\beta$ is\cite{linalgDavid}
        \begin{equation}
            A =
                \begin{pmatrix}
                    \prd{\alpha}[\ecp{E_L}][2] &
                    \frac{\prtl^2\ecp{E_L}}{\prtl\alpha\prtl\beta} \\
                    \frac{\prtl^2\ecp{E_L}}{\prtl\beta\prtl\alpha} &
                    \prd{\beta}[\ecp{E_L}][2]\\
                \end{pmatrix}
            \label{eq:analytichessen}
        \end{equation}
    We solve the derivatives generally for a variatonal parameters $c_n$. The
    first derivative is thus
        \begin{align}
            \prd{c_n}[\ecp{E_L}] &=
            \prd{c_n}\left(\int\frac{\psi_T^2E_L}{\psinorm}\right) \nonumber \\
            &= \int\left(\frac{\psinorm\left(2\psi_TE_L\psigender{n} +
            \psi_T^2\prd{c_n}[E_L]\right) -
            2\psi_T^2E_L\int\psi_T\psigender{n}\md\tau}
            {\left(\psinorm\right)^2}\right) \md\tau \nonumber \\
            &= 2\ecp{\frac{E_L}{\psi_T}\psigender{n}} -
            2\ecp{E_L}\ecp{\frac{1}{\psi_T}\psigender{n}}
            \label{eq:deralpahecpEl}
        \end{align}
    In the last step the hermiticity of the Hamiltonian was used. We have also
    used the fact that the trial wave function $\psi_T$ is a real function. \\
    The second derivative elements is then
        \begin{align}
            \mixprd{\ecp{E_L}}[c_n][c_m] &= 2\left(\prd{c_m}
            \left(\ecp{\frac{E_L}{\psi_T}\psigender{n}}\right) -
            \ecp{E_L}\prd{c_m} \left(\ecp{\frac{1}{\psi_T}\psigender{n}}\right)
            - \ecp{\frac{1}{\psi_T}\psigender{n}} \prd{c_m}[\ecp{E_L}]\right)
            \nonumber \\
            &\begin{aligned}
                \hspace{0.04cm}=&\hspace{0.07cm} 2\left(\prd{c_m}
                \ecp{\frac{E_L}{\psi_T}\psigender{n}}\right) -
                \ecp{E_L}\prd{c_m}
                \left(\ecp{\frac{1}{\psi_T}\psigender{n}}\right) \\
                &- 4\ecp{\frac{1}{\psi_T}\psigender{n}}
                \left(\ecp{\frac{E_L}{\psi_T}\psigender{m}} -
                \ecp{E_L}\ecp{\frac{1}{\psi_T}\psigender{m}}\right)
            \end{aligned} \nonumber \\
            &\begin{aligned}
                \hspace{0.04cm}=&\hspace{0.07cm}
                2\left(\ecp{\frac{E_L}{\psi^2_T} \psigender{n} \psigender{m}} +
                \ecp{\frac{1}{\psi_T} \prd{c_m}[E_L] \psigender{n}}  +
                \ecp{\frac{E_L}{\psi_T} \mixprd{\psi_T}[c_n][c_m]} -
                2\ecp{\frac{E_L}{\psi_T} \psigender{n}} \ecp{\frac{1}{\psi_T}
                \psigender{m}} \right) \\
                &- 2\ecp{E_L} \left(\ecp{\frac{1}{\psi^2_T} \psigender{n}
                \psigender{m}} + \ecp{\frac{1}{\psi_T}
                \mixprd{\psi_T}[c_n][c_m]} - 2\ecp{\frac{1}{\psi_T}
                \psigender{n}} \ecp{\frac{1}{\psi_T} \psigender{m}}\right) \\
                &- 4\left(\ecp{\frac{1}{\psi_T} \psigender{n}} \ecp{\frac{E_L}{\psi_T}
                \psigender{m}} - \ecp{E_L} \ecp{\frac{1}{\psi_T}
                \psigender{n}} \ecp{\frac{1}{\psi_T} \psigender{m}}\right)
            \end{aligned} \nonumber \\
            &\begin{aligned}
                \hspace{0.04cm}=&\hspace{0.07cm} 2\left(\ecp{\frac{E_L}{\psi_T}
                \mixprd{\psi_T}[c_n][c_m]} - \ecp{E_L}\ecp{\frac{1}{\psi_T}
                \mixprd{\psi_T}[c_n][c_m]} + \ecp{\frac{E_L}{\psi^2_T}
                \psigender{n} \psigender{m}} - \ecp{E_L}\ecp{\frac{1}{\psi^2_T}
                \psigender{n} \psigender{m}}\right) \\
                &+ \ecp{E_L} \ecp{\frac{1}{\psi_T}\psigender{n}}
                \ecp{\frac{1}{\psi}\psigender{m}} - 2\left(
                \ecp{\frac{E_L}{\psi_T}\psigender{n}}
                \ecp{\frac{1}{\psi_T}\psigender{m}} + \ecp{\frac{E_L}{\psi_T}
                \psigender{m}} \ecp{\frac{1}{\psi_T} \psigender{n}}\right) \\
                &+ 2\ecp{\frac{1}{\psi_T} \psigender{n} \prd{c_m}[E_L]}
            \end{aligned}
            \label{eq:der2alpahecpElfir}
        \end{align}
\newcommand{\Hij}[3]{H_{{#2}_{{#1}_j}#3}\left(\sqrt{\alpha\omega}{#1}_i\right)}
\newcommand{\eri}{\exp(-\frac{\alpha\omega}{2}r^2_i)}
    For the derivative of the wavefunction, we find the elements individually
    by using Jacobi's formula as described in
    \Autoref{sub:derivative_of_determinant}. We see that we need to first find
    the derivative to the individual elements in $\Phi$. These are found by
    using the product rule, but we also see that we need the derivative to the
    individual Hermite polynomials. These are simply (with the  chain rule)
        \begin{align}
            \prd{x_i}\Hij{x}{n}{} &=
            \frac{x_i}{2}\sqrt{\frac{\omega}{\alpha}}\prd{x_i}[H_{n_{x_j}}]
            \nonumber \\
            &= n_{x_j}x_i\sqrt{\frac{\omega}{\alpha}}\Hij{x}{n}{-1} \nonumber
            \\
            &= \frac{n_{x_j}}{2\alpha}\left(\sqrt{\alpha} +
            2x_i\sqrt{\omega}\left(n_{x_j}-1\right)
            \frac{\Hij{x}{n}{-2}}{\Hij{x}{n}{}}\right)\Hij{x}{n}{}
            \label{eq:phideralph1x}
        \end{align}
    The same differentiation is valid for $H_{n_{y_j}}$ as well. In the latter
    step we used \Autoref{eq:hermiteReq}. The first derivative of $\Phi_{ij}$
    with respect to $\alpha$ is thus
        \begin{align}
            \prd{\alpha}[\Phi_{ij}] &= \prd{\alpha} \left(\Hij{x}{n}{}\Hij{y}{n}{}
            \exp(-\frac{\alpha\omega}{2}r^2_i)\right) \nonumber \\
            &= \left(\Hij{y}{n}{}\prd{\alpha}\Hij{x}{n}{} +
            \Hij{x}{n}{}\prd{\alpha}\Hij{y}{n}{} - \frac{\omega
            r^2_i}{2}\right)\eri \nonumber \\
            &= \frac{1}{2}\left(\Theta_{ij}(\alpha,x) + \Theta_{ij}(\alpha,y) -
            \omega r^2_i\right)\Phi_{ij}
            \label{eq:phideralpha1}
        \end{align}
    with
        \begin{equation}
            \Theta_{ij}(\alpha,x) \equiv
            \frac{n_{x_j}}{\alpha}\left(\sqrt{\alpha} +
            2x_i\sqrt{\omega}\left(n_{x_j}-1\right)
            \frac{\Hij{x}{n}{-2}}{\Hij{x}{n}{}}\right)
            \label{eq:Thetadef}
        \end{equation}
    Using Jacobi's formula the first derivative with respect to $\alpha$ of
    $\psi_T$ is
        \begin{align}
            \prd{\alpha}[\psi_T] &= g\left(\vec{r},\beta\right)
            \prd{\alpha}\left(\det(\Phi)\right) \nonumber \\
            &= g\left(\vec{r},\beta\right) \Tr(\adj{\Phi}\prd{\alpha}[\Phi])
            \nonumber \\
            &= 
            g\left(\vec{r},\beta\right)
            \det(\Phi)\suml{i,j}{N} \Phi^{-1}_{ij}\prd{\alpha}[\Phi_{ji}]
            \nonumber \\
            &= \psi_T \suml{i,j}{N} \Phi^{-1}_{ij}
            \Phi_{ji}\left(\left(\Theta_{ji}(\alpha,x) + \Theta_{ji}(\alpha,y)\right) -
            \omega r^2_j\right)
            \label{eq:psideralpha}
        \end{align}
    where we have used \Autoref{eq:aadja} and defined $\Theta$ to be the vector
    of element as defined in \Autoref{eq:Thetadef} \\
    The second derivative of $\psi_T$ with respect to $\alpha$ we need the
    second derivative of the Hermite polynomials. These are
        \begin{align}
            \prd{\alpha}[][2]\Hij{x}{n}{} &= \prd{\alpha} \left(n_{x_j}x_i
            \sqrt{\frac{\omega}{\alpha}}\Hij{x}{n}{-1}\right)
            \nonumber \\
            &= n_{x_j}\left(n_{x_j}-1\right)x^2_i
            \frac{\omega}{\alpha}\Hij{x}{n}{-2} -\frac{n_{x_j}x_i}{2}
            \sqrt{\frac{\omega}{\alpha^3}}\Hij{x}{n}{-1} \nonumber \\
            &= \frac{n_{x_j}}{\alpha} \left(\left(n_{x_j}-1\right)x_i
            \left(\omega - \frac{1}{4}\sqrt{\frac{\omega}{\alpha}}\right)
            \frac{\Hij{x}{n}{-2}}{\Hij{x}{n}{}} -
            \frac{n_{x_j}}{4\alpha^2}\right)\Hij{x}{n}{}
            \label{eq:Hderalphasec}
        \end{align}
    The derivation is the same for $H_{n_{y_j}}$. The second derivative of
    $\Phi_{ij}$ is thus
        \begin{align}
            \prd{\alpha}[\phi_{ij}][2] &= \prd{\alpha}\left[
                \left(\Hij{y}{n}{}\prd{\alpha}\Hij{x}{n}{} +
                \Hij{x}{n}{}\prd{\alpha}\Hij{y}{n}{} - \frac{\omega
                r^2_i}{2}\right)\eri\right] \nonumber \\
            &\hspace{-0.04cm}\begin{aligned}
                &= \left(2\prd{\alpha}[\Hij{x}{n}{}]\prd{\alpha}[\Hij{y}{n}{}]
                \Hij{y}{n}{}\prd{\alpha}[\Hij{x}{n}{}][2] \right.\\ 
                &\hspace{0.6cm} + \left.
                \Hij{x}{n}{}\prd{\alpha}[\Hij{y}{n}{}][2] -
                \prd{\alpha}[\Phi_{ij}] \frac{\omega r^2_i}{2}\right)
                \exp(-\frac{\alpha\omega}{2}r^2_i)
            \end{aligned}
            &=
            \label{eq:phideralphasec}
        \end{align}
    The derivative with respect to $\beta$ follows a similar approach as in
    \Autoref{ssub:general_case} giving
        \begin{align}
            \prd{\beta}[\psi_T] &= \det(\Phi\left(\vec{r},\alpha\right))
            \prd{\beta}\left(\prod_{i<j}\gJ\right) \nonumber \\
            &= \det(\Phi\left(\vec{r},\alpha\right))
            g\left(\vec{r},\beta\right)
            \prd{\beta}\left(\sum_{i<j}\frac{a}{\beta +
            \frac{1}{r_{ij}}}\right) \nonumber \\
            &= -\psi_T \sum_{i\neq j} \frac{a}{\left(\beta +
            \frac{1}{r_{ij}}\right)^2}
            \label{eq:psiderbeta}
        \end{align}
    The second derivative is thus
        \begin{align}
            \prd{\beta}[\psi_T][2] &= \prd{\beta}\left(-\psi_T \sum_{i\neq j}
            \frac{a}{\left(\beta + \frac{1}{r_{ij}}\right)^2}\right) \nonumber
            \\
            &= \left(\left(\sum\limits_{i\neq j}\frac{a}{\left(\beta +
            \frac{1}{r_{ij}}\right)^2}\right)^2 + \sum_{i\neq
            j}\frac{2a}{\left(\beta + \frac{1}{r_{ij}}\right)^3} \right)\psi_T
            \label{eq:psiderbetasec}
        \end{align}
    where we simply used the product rule for differentiation. \\
    The mixed second derivative is found with
    \Autoref{eq:psideralpha,eq:psiderbeta} giving
        \begin{align}
            \frac{\prtl^2\psi_T}{\prtl\alpha\beta} &=
            \prd{\beta}\left(-\psi_T\frac{\omega}{2}\suml{i}{N}r^2_i\right)
            \nonumber \\
            &= \psi_T\frac{\omega}{2}\suml{i}{N}r^2_i\sum_{i\neq
            j}\frac{1}{\left(\beta + \frac{1}{r_{ij}}\right)^2}
            \label{eq:psideralpbetsec}
        \end{align}
    The final expression for the elements in the Hessen matrix is thus
        \begin{equation}
            \begin{aligned}
                \prd{\alpha}[\ecp{E_L}][2] &= \omega^2 \left(\ecp{E_LR^2} -
                \ecp{E_L}\ecp{R^2} + \ecp{E_L}\ecp{R}^2 -
                \ecp{E_LR}\ecp{R}\right) - \omega\ecp{R\prd{\alpha}[E_L]} \\
                \prd{\beta}[\ecp{E_L}][2] &= 4\left(\ecp{E_LB^2_2} -
                \ecp{E_L}\ecp{B^2_2}\right) + \ecp{E_L}\ecp{B_2}^2 -
                \ecp{E_LB_2}\ecp{B_2} - 2\ecp{B_2\prd{\beta}[E_L]} \\
                \mixprd{\ecp{E_L}}[\alpha][\beta] &=
                \omega\left(2\left(\ecp{E_LRB_2} - \ecp{E_L}\ecp{RB_2}\right) +
                \frac{1}{2}\ecp{E_L}\ecp{R}\ecp{B_2} - \ecp{E_LR}\ecp{B_2} -
                \ecp{E_LB_2}\ecp{R} - \ecp{R\prd{\beta}[E_L]}\right)
            \end{aligned}
            \label{eq:HessenElemFin}
        \end{equation}
    where 
        \begin{equation}
            \begin{aligned}
                R &\equiv \suml{i}{N}r^2_i \\ 
                B_n &\equiv \sum\limits_{i\neq j} \frac{a}{\left(\beta +
                \frac{1}{r_{ij}}\right)^n} 
            \end{aligned}
            \hspace{0.5cm}
            \begin{aligned}
                \prd{\alpha}[\psi_T] &= -\frac{\omega}{2}R\psi_T  \\
                \prd{\beta}[\psi_T] &= -B_2\psi_T 
            \end{aligned}
            \hspace{0.5cm}
            \begin{aligned}
                \prd{\alpha}[\psi_T][2] &= \frac{\omega^2}{4}R^2\psi_T \\
                \prd{\beta}[\psi_T][2] &= \left(B^2_2 + 2B_3\right)\psi_T
            \end{aligned}
            \hspace{0.5cm}
            \begin{aligned}
                \frac{\prtl^2\psi_T}{\prtl\alpha\prtl\beta} =
                \frac{\omega}{2}RB_2\psi_T
            \end{aligned}
            \label{eq:RBdef}
        \end{equation}

\end{document}
